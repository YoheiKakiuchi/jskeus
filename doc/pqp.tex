\section{干渉計算}

干渉計算には2組の幾何モデルが交差するかを判定する物である.
irteusではノースカロライナ大学のLin氏らのグループにより開発されたPQPを
他言語インターフェースを介して利用できるようにしてある.
(他の干渉計算ソフトウェアパッケージについてはhttp://gamma.cs.unc.edu/research/collision/に詳しい.)
PQPは
(1)2つのモデルが交差するかを判定する衝突検出,
(2)2つのモデル間の最初距離を算出する距離計算,
(3)2つのモデルがある距離以下であるかを判定する近接検証,
等の3つ機能を提供する.

PQPソフトウェアパッケージの使い方はirteus/PQP/README.txtに
書いてあり,irteus/PQP/src/PQP.hを読むことで理解できるようになって
いる.

\subsection{irteusからPQPの呼び出し}

irteusでPQPを使うためのファイルは
CPQP.C, euspqp.c, pqp.l
からなる.
2つの幾何モデルが衝突してしるか否かを判定するためには,

{\baselineskip=10pt
\begin{verbatim}
(defun pqp-collision-check (model1 model2
				       &optional (flag PQP_FIRST_CONTACT) &key (fat 0) (fat2 nil))
  (let ((m1 (get model1 :pqpmodel))  (m2 (get model2 :pqpmodel))
        (r1 (send model1 :worldrot)) (t1 (send model1 :worldpos))
        (r2 (send model2 :worldrot)) (t2 (send model2 :worldpos)))
    (if (null fat2) (setq fat2 fat))
    (if (null m1) (setq m1 (send model1 :make-pqpmodel :fat fat)))
    (if (null m2) (setq m2 (send model2 :make-pqpmodel :fat fat2)))
    (pqpcollide r1 t1 m1 r2 t2 m2 flag)))
\end{verbatim}
}
を呼び出せば良い.
r1,r1,r2,t1はそれぞれの物体の並進ベクトル,回転行列となり,
(get model1 :pqpmodel)でPQPの幾何モデルへのポインタを参照する.
このポインタは:make-pqpmodelメソッドの中で以下のよう計算される.
{\baselineskip=10pt
\begin{verbatim}
(defmethod cascaded-coords
  (:make-pqpmodel
   (&key (fat 0))
   (let ((m (pqpmakemodel))
         vs v1 v2 v3 (id 0))
     (setf (get self :pqpmodel) m)
     (pqpbeginmodel m)
     (dolist (f (send self :faces))
       (dolist (poly (face-to-triangle-aux f))
         (setq vs (send poly :vertices)
               v1 (send self :inverse-transform-vector (first vs))
               v2 (send self :inverse-transform-vector (second vs))
               v3 (send self :inverse-transform-vector (third vs)))
         (when (not (= fat 0))
           (setq v1 (v+ v1 (scale fat (normalize-vector v1)))
                 v2 (v+ v2 (scale fat (normalize-vector v2)))
                 v3 (v+ v3 (scale fat (normalize-vector v3)))))
         (pqpaddtri m v1 v2 v3 id)
         (incf id)))
     (pqpendmodel m)
     m)))
\end{verbatim}
}
ここでは,まず(pqpmakemodel)が呼び出されている.
pqpmakemodelの中では,euqpqp.cで定義されている,

{\baselineskip=10pt
\begin{verbatim}
pointer PQPMAKEMODEL(register context *ctx, int n, register pointer *argv)
{
    int addr = PQP_MakeModel();
    return makeint(addr);
}
\end{verbatim}
}

が呼び出されており,これは,CPQP.Cの
{\baselineskip=10pt
\begin{verbatim}
PQP\_Model *PQP_MakeModel()
{
    return new PQP_Model();
}
\end{verbatim}
}
が呼ばれている.PQP\_Model()はPQP.hで定義されているものであり,
この様にしてeuslisp内の関数が実際のPQPライブラリの関数に渡されてい
る以降,(pqpbeginmodel m)でPQPの幾何モデルのインスタンスを作成し,
(pqpaddtri m v1 v2 v3 id)として面情報を登録している.

\subsubsection{物体形状モデル同士の干渉計算例}
pqp-collision-checkやpqp-collision-distanceを利用した例を示す.
{\baselineskip=10pt
\begin{verbatim}
;; Make models
(setq *b0* (make-cube 100 100 100))
(setq *b1* (make-cube 100 100 100))

;; Case 1 : no collision
(send *b0* :newcoords (make-coords :pos #f(100 100 -100)
                                   :rpy (list (deg2rad 10) (deg2rad -20) (deg2rad 30))))
(objects (list *b0* *b1*))
(print (pqp-collision-check *b0* *b1*)) ;; Check collision
(let ((ret (pqp-collision-distance *b0* *b1*))) ;; Check distance and nearest points
  (print (car ret)) ;; distance
  (send (cadr ret) :draw-on :flush nil :size 20 :color #f(1 0 0)) ;; nearest point on *b0*
  (send (caddr ret) :draw-on :flush nil :size 20 :color #f(1 0 0)) ;; nearest point on *b1*
  (send *irtviewer* :viewer :draw-line (cadr ret) (caddr ret))
  (send *irtviewer* :viewer :viewsurface :flush))

;; Case 2 : collision
(send *b0* :newcoords (make-coords :pos #f(50 50 -50)
                                   :rpy (list (deg2rad 10) (deg2rad -20) (deg2rad 30))))
(objects (list *b0* *b1*))
(print (pqp-collision-check *b0* *b1*)) ;; Check collision
(let ((ret (pqp-collision-distance *b0* *b1*))) ;; Check distance and nearest points
  (print (car ret)) ;; distance
  ;; In case of collision, nearest points are insignificant values.
  (send (cadr ret) :draw-on :flush nil :size 20 :color #f(1 0 0)) ;; nearest point on *b0*
  (send (caddr ret) :draw-on :flush nil :size 20 :color #f(1 0 0)) ;; nearest point on *b1*
  (send *irtviewer* :viewer :draw-line (cadr ret) (caddr ret))
  (send *irtviewer* :viewer :viewsurface :flush))
\end{verbatim}
}

\subsection{ロボット動作と干渉計算}

ハンドで物体をつかむ,という動作の静的なシミュレーションを行う場合に
手(指)のリンクと対象物体の干渉を調べ,これが起こるところで物体をつか
む動作を停止させるということが出来る.

{\baselineskip=10pt
\begin{verbatim}
(objects (list *sarm* *target*))

(send *sarm* :solve-ik *target* :debug-view t)
(while (> a 0)
  (if (pqp-collision-check-objects
       (list (send *sarm* :joint-fr :child-link)
             (send *sarm* :joint-fl :child-link))
       (list *target*))
      (return))
  (decf a 0.1)
  (send *irtviewer* :draw-objects)
  (send *sarm* :move-fingers a))
(send *sarm* :end-coords :assoc *target*)

(dotimes (i 100)
  (send *sarm* :joint0 :joint-angle 1 :relative t)
  (send *irtviewer* :draw-objects))
(send *sarm* :end-coords :dissoc *target*)
(dotimes (i 100)
  (send *sarm* :joint0 :joint-angle -1 :relative t)
  (send *irtviewer* :draw-objects))
\end{verbatim}
}

同様の機能が,"irteus/demo/sample-arm-model.l"ファイルの:open-hand,
:close-handというメソッドで提供されている.

 \input{pqp-func}
